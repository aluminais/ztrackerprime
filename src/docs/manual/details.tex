%/*************************************************************************
% *
% * FILE  details.tex
% * $Id: details.tex,v 1.4 2001/09/08 13:59:36 tlr Exp $
% *
% * DESCRIPTION 
% *   The details appendix.
% *
% * This file is part of ztracker - a tracker-style MIDI sequencer.
% *
% * Copyright (c) 2001, Daniel Kahlin <tlr@users.sourceforge.net>
% * All rights reserved.
% *
% * Redistribution and use in source and binary forms, with or without
% * modification, are permitted provided that the following conditions
% * are met:
% * 1. Redistributions of source code must retain the above copyright
% *    notice, this list of conditions and the following disclaimer.
% * 2. Redistributions in binary form must reproduce the above copyright
% *    notice, this list of conditions and the following disclaimer in the
% *    documentation and/or other materials provided with the distribution.
% * 3. Neither the names of the copyright holders nor the names of their
% *    contributors may be used to endorse or promote products derived 
% *    from this software without specific prior written permission.
% *
% * THIS SOFTWARE IS PROVIDED BY THE COPYRIGHT HOLDERS AND CONTRIBUTORS
% * ``AS IS�� AND ANY EXPRESS OR IMPLIED WARRANTIES, INCLUDING, BUT NOT
% * LIMITED TO, THE IMPLIED WARRANTIES OF MERCHANTABILITY AND FITNESS FOR
% * A PARTICULAR PURPOSE ARE DISCLAIMED.  IN NO EVENT SHALL THE REGENTS OR
% * CONTRIBUTORS BE LIABLE FOR ANY DIRECT, INDIRECT, INCIDENTAL, SPECIAL,
% * EXEMPLARY, OR CONSEQUENTIAL DAMAGES (INCLUDING, BUT NOT LIMITED TO,
% * PROCUREMENT OF SUBSTITUTE GOODS OR SERVICES; LOSS OF USE, DATA, OR
% * PROFITS; OR BUSINESS INTERRUPTION) HOWEVER CAUSED AND ON ANY THEORY OF
% * LIABILITY, WHETHER IN CONTRACT, STRICT LIABILITY, OR TORT (INCLUDING
% * NEGLIGENCE OR OTHERWISE) ARISING IN ANY WAY OUT OF THE USE OF THIS
% * SOFTWARE, EVEN IF ADVISED OF THE POSSIBILITY OF SUCH DAMAGE.
% *
% ******/
\chapter{Details}

% processing order

\section{Processing Order}
  Tracks are processed in ascending order.  Any commands/data in track 1
is processed and sent before commands/data in track 2, and so on\ldots{}
Each track has its own parameter memory.  For example, if you play the same
midi channel from both track 1 and 2 and do a pitch slide in track 1, when you
try continue the pitch slide from track 2, it will start from the same point
that track 1 started from.


% pitch bend values

\section{Pitch Bend Values}
  Pitch effects are ways of setting the MIDI Pitch bend value.  Normally a synth has
a pitch bend \emph{range} defined for each sound.  This \emph{range} determines how
many semitones the pitch is shifted for the maximum pitch bend \emph{value}, up or down.
Mosts synths allow you to set the range to between 0 and 12, some even up to 24
(two octaves)

% pitch bend formulae
The pitch bend value corresponding to a particular number of semitones pitch shift can
be calculated by the following formula:

\begin{equation}
value=round\biggl( 8192 + \frac{8192}{range} \cdot \Delta P_{semi} \biggr)
\end{equation}

Or if cents are preferred:

\begin{equation}
value=round\biggl( 8192 + \frac{8192}{range \cdot 100} \cdot \Delta P_{cents} \biggr)
\end{equation}

Note that these formulae gives values from 0 to 16384, and that pitch bend values
range from 0 to 16383.  To solve this 16384 is replaced with 16383, the error
introduced is unnoticable. 

% Pitch slides
\section{Pitch Slides}
  Pitch slides are instructions to the player to update the pitch bend value every
subtick (which is 24 ppqn). $ticks$ is the duration of the slide in number of rows.
$tpb$ is the number of ticks per beat. $\Delta_{value}$ is the desired resulting pitch bend
change after $ticks$ ticks have elapsed.  $slide$ is the parameter to use with the
slide command.
\begin{equation}
slide=round\biggl( \Delta_{value} \cdot \frac{tpb}{subticks \cdot ticks} \biggr)
\end{equation}
This equation can then be merged with the previous pitch formulae.
\begin{equation}
slide=round\Biggl( \biggl( \frac{8192}{range} \cdot \Delta P_{semi} \biggr) \cdot \frac{tpb}{24 \cdot ticks} \Biggr)
\end{equation}
\begin{equation}
slide=round\Biggl( \biggl( \frac{8192}{range \cdot 100} \cdot \Delta P_{cents} \biggr) \cdot \frac{tpb}{24 \cdot ticks} \Biggr)
\end{equation}


% pitchbend tables
%
% Do not edit this file!
% It was automatically generated by pitchtables.pl.
%
\begin{table}
\begin{scriptsize}
\begin{tabular}{|l|l|l|l|l|l|l|l|l|l|l|l|}
\hline
12 & 11 & 10 & 9 & 8 & 7 & 6 & 5 & 4 & 3 & 2 & 1 \\
\hline
\pc{3FFF} & \pc{3D55} & \pc{3AAB} & \pc{3800} & \pc{3555} & \pc{32AB} & \pc{3000} & \pc{2D55} & \pc{2AAB} & \pc{2800} & \pc{2555} & \pc{22AB} \\
\hline\hline\hline
-1 & -2 & -3 & -4 & -5 & -6 & -7 & -8 & -9 & -10 & -11 & -12 \\
\hline
\pc{1D55} & \pc{1AAB} & \pc{1800} & \pc{1555} & \pc{12AB} & \pc{1000} & \pc{0D55} & \pc{0AAB} & \pc{0800} & \pc{0555} & \pc{02AB} & \pc{0000} \\
\hline
\end{tabular}
\end{scriptsize}
\caption[Pitchbend Values for Bend Range 12]{Pitchbend Values for Bend Range 12}
\end{table}
\begin{table}
\begin{scriptsize}
\begin{tabular}{|l|l|l|l|l|l|l|l|l|l|l|l|}
\hline
24 & 23 & 22 & 21 & 20 & 19 & 18 & 17 & 16 & 15 & 14 & 13 \\
\hline
\pc{3FFF} & \pc{3EAB} & \pc{3D55} & \pc{3C00} & \pc{3AAB} & \pc{3955} & \pc{3800} & \pc{36AB} & \pc{3555} & \pc{3400} & \pc{32AB} & \pc{3155} \\
\hline\hline
12 & 11 & 10 & 9 & 8 & 7 & 6 & 5 & 4 & 3 & 2 & 1 \\
\hline
\pc{3000} & \pc{2EAB} & \pc{2D55} & \pc{2C00} & \pc{2AAB} & \pc{2955} & \pc{2800} & \pc{26AB} & \pc{2555} & \pc{2400} & \pc{22AB} & \pc{2155} \\
\hline\hline\hline
-1 & -2 & -3 & -4 & -5 & -6 & -7 & -8 & -9 & -10 & -11 & -12 \\
\hline
\pc{1EAB} & \pc{1D55} & \pc{1C00} & \pc{1AAB} & \pc{1955} & \pc{1800} & \pc{16AB} & \pc{1555} & \pc{1400} & \pc{12AB} & \pc{1155} & \pc{1000} \\
\hline\hline
-13 & -14 & -15 & -16 & -17 & -18 & -19 & -20 & -21 & -22 & -23 & -24 \\
\hline
\pc{0EAB} & \pc{0D55} & \pc{0C00} & \pc{0AAB} & \pc{0955} & \pc{0800} & \pc{06AB} & \pc{0555} & \pc{0400} & \pc{02AB} & \pc{0155} & \pc{0000} \\
\hline
\end{tabular}
\end{scriptsize}
\caption[Pitchbend Values for Bend Range 24]{Pitchbend Values for Bend Range 24}
\end{table}
\begin{table}
\begin{scriptsize}
\begin{tabular}{|l|l|l|l|l|l|l|l|l|l|l|l|l|}
\hline
semi & 12 & 11 & 10 & 9 & 8 & 7 & 6 & 5 & 4 & 3 & 2 & 1 \\
\hline
12 & \pc{3FFF} & \pc{----} & \pc{----} & \pc{----} & \pc{----} & \pc{----} & \pc{----} & \pc{----} & \pc{----} & \pc{----} & \pc{----} & \pc{----} \\
\hline
11 & \pc{3D55} & \pc{3FFF} & \pc{----} & \pc{----} & \pc{----} & \pc{----} & \pc{----} & \pc{----} & \pc{----} & \pc{----} & \pc{----} & \pc{----} \\
\hline
10 & \pc{3AAB} & \pc{3D17} & \pc{3FFF} & \pc{----} & \pc{----} & \pc{----} & \pc{----} & \pc{----} & \pc{----} & \pc{----} & \pc{----} & \pc{----} \\
\hline
9 & \pc{3800} & \pc{3A2F} & \pc{3CCD} & \pc{3FFF} & \pc{----} & \pc{----} & \pc{----} & \pc{----} & \pc{----} & \pc{----} & \pc{----} & \pc{----} \\
\hline
8 & \pc{3555} & \pc{3746} & \pc{399A} & \pc{3C72} & \pc{3FFF} & \pc{----} & \pc{----} & \pc{----} & \pc{----} & \pc{----} & \pc{----} & \pc{----} \\
\hline
7 & \pc{32AB} & \pc{345D} & \pc{3666} & \pc{38E4} & \pc{3C00} & \pc{3FFF} & \pc{----} & \pc{----} & \pc{----} & \pc{----} & \pc{----} & \pc{----} \\
\hline
6 & \pc{3000} & \pc{3174} & \pc{3333} & \pc{3555} & \pc{3800} & \pc{3B6E} & \pc{3FFF} & \pc{----} & \pc{----} & \pc{----} & \pc{----} & \pc{----} \\
\hline
5 & \pc{2D55} & \pc{2E8C} & \pc{3000} & \pc{31C7} & \pc{3400} & \pc{36DB} & \pc{3AAB} & \pc{3FFF} & \pc{----} & \pc{----} & \pc{----} & \pc{----} \\
\hline
4 & \pc{2AAB} & \pc{2BA3} & \pc{2CCD} & \pc{2E39} & \pc{3000} & \pc{3249} & \pc{3555} & \pc{399A} & \pc{3FFF} & \pc{----} & \pc{----} & \pc{----} \\
\hline
3 & \pc{2800} & \pc{28BA} & \pc{299A} & \pc{2AAB} & \pc{2C00} & \pc{2DB7} & \pc{3000} & \pc{3333} & \pc{3800} & \pc{3FFF} & \pc{----} & \pc{----} \\
\hline
2 & \pc{2555} & \pc{25D1} & \pc{2666} & \pc{271C} & \pc{2800} & \pc{2925} & \pc{2AAB} & \pc{2CCD} & \pc{3000} & \pc{3555} & \pc{3FFF} & \pc{----} \\
\hline
1 & \pc{22AB} & \pc{22E9} & \pc{2333} & \pc{238E} & \pc{2400} & \pc{2492} & \pc{2555} & \pc{2666} & \pc{2800} & \pc{2AAB} & \pc{3000} & \pc{3FFF} \\
\hline
0 & \pc{2000} & \pc{2000} & \pc{2000} & \pc{2000} & \pc{2000} & \pc{2000} & \pc{2000} & \pc{2000} & \pc{2000} & \pc{2000} & \pc{2000} & \pc{2000} \\
\hline
-1 & \pc{1D55} & \pc{1D17} & \pc{1CCD} & \pc{1C72} & \pc{1C00} & \pc{1B6E} & \pc{1AAB} & \pc{199A} & \pc{1800} & \pc{1555} & \pc{1000} & \pc{0000} \\
\hline
-2 & \pc{1AAB} & \pc{1A2F} & \pc{199A} & \pc{18E4} & \pc{1800} & \pc{16DB} & \pc{1555} & \pc{1333} & \pc{1000} & \pc{0AAB} & \pc{0000} & \pc{----} \\
\hline
-3 & \pc{1800} & \pc{1746} & \pc{1666} & \pc{1555} & \pc{1400} & \pc{1249} & \pc{1000} & \pc{0CCD} & \pc{0800} & \pc{0000} & \pc{----} & \pc{----} \\
\hline
-4 & \pc{1555} & \pc{145D} & \pc{1333} & \pc{11C7} & \pc{1000} & \pc{0DB7} & \pc{0AAB} & \pc{0666} & \pc{0000} & \pc{----} & \pc{----} & \pc{----} \\
\hline
-5 & \pc{12AB} & \pc{1174} & \pc{1000} & \pc{0E39} & \pc{0C00} & \pc{0925} & \pc{0555} & \pc{0000} & \pc{----} & \pc{----} & \pc{----} & \pc{----} \\
\hline
-6 & \pc{1000} & \pc{0E8C} & \pc{0CCD} & \pc{0AAB} & \pc{0800} & \pc{0492} & \pc{0000} & \pc{----} & \pc{----} & \pc{----} & \pc{----} & \pc{----} \\
\hline
-7 & \pc{0D55} & \pc{0BA3} & \pc{099A} & \pc{071C} & \pc{0400} & \pc{0000} & \pc{----} & \pc{----} & \pc{----} & \pc{----} & \pc{----} & \pc{----} \\
\hline
-8 & \pc{0AAB} & \pc{08BA} & \pc{0666} & \pc{038E} & \pc{0000} & \pc{----} & \pc{----} & \pc{----} & \pc{----} & \pc{----} & \pc{----} & \pc{----} \\
\hline
-9 & \pc{0800} & \pc{05D1} & \pc{0333} & \pc{0000} & \pc{----} & \pc{----} & \pc{----} & \pc{----} & \pc{----} & \pc{----} & \pc{----} & \pc{----} \\
\hline
-10 & \pc{0555} & \pc{02E9} & \pc{0000} & \pc{----} & \pc{----} & \pc{----} & \pc{----} & \pc{----} & \pc{----} & \pc{----} & \pc{----} & \pc{----} \\
\hline
-11 & \pc{02AB} & \pc{0000} & \pc{----} & \pc{----} & \pc{----} & \pc{----} & \pc{----} & \pc{----} & \pc{----} & \pc{----} & \pc{----} & \pc{----} \\
\hline
-12 & \pc{0000} & \pc{----} & \pc{----} & \pc{----} & \pc{----} & \pc{----} & \pc{----} & \pc{----} & \pc{----} & \pc{----} & \pc{----} & \pc{----} \\
\hline
\end{tabular}
\end{scriptsize}
\caption[Pitchbend Values for Bend Ranges 1-12]{Pitchbend Values for Bend Ranges 1-12}
\end{table}
% eof


% eof
