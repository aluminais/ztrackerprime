%/*************************************************************************
% *
% * FILE  gettingstarted.tex
% * $Id: gettingstarted.tex,v 1.2 2001/09/10 21:07:13 tlr Exp $
% *
% * DESCRIPTION 
% *   The getting started using ztracker chapter.
% *
% * This file is part of ztracker - a tracker-style MIDI sequencer.
% *
% * Copyright (c) 2001, Daniel Kahlin <tlr@users.sourceforge.net>
% * All rights reserved.
% *
% * Redistribution and use in source and binary forms, with or without
% * modification, are permitted provided that the following conditions
% * are met:
% * 1. Redistributions of source code must retain the above copyright
% *    notice, this list of conditions and the following disclaimer.
% * 2. Redistributions in binary form must reproduce the above copyright
% *    notice, this list of conditions and the following disclaimer in the
% *    documentation and/or other materials provided with the distribution.
% * 3. Neither the names of the copyright holders nor the names of their
% *    contributors may be used to endorse or promote products derived 
% *    from this software without specific prior written permission.
% *
% * THIS SOFTWARE IS PROVIDED BY THE COPYRIGHT HOLDERS AND CONTRIBUTORS
% * ``AS IS�� AND ANY EXPRESS OR IMPLIED WARRANTIES, INCLUDING, BUT NOT
% * LIMITED TO, THE IMPLIED WARRANTIES OF MERCHANTABILITY AND FITNESS FOR
% * A PARTICULAR PURPOSE ARE DISCLAIMED.  IN NO EVENT SHALL THE REGENTS OR
% * CONTRIBUTORS BE LIABLE FOR ANY DIRECT, INDIRECT, INCIDENTAL, SPECIAL,
% * EXEMPLARY, OR CONSEQUENTIAL DAMAGES (INCLUDING, BUT NOT LIMITED TO,
% * PROCUREMENT OF SUBSTITUTE GOODS OR SERVICES; LOSS OF USE, DATA, OR
% * PROFITS; OR BUSINESS INTERRUPTION) HOWEVER CAUSED AND ON ANY THEORY OF
% * LIABILITY, WHETHER IN CONTRACT, STRICT LIABILITY, OR TORT (INCLUDING
% * NEGLIGENCE OR OTHERWISE) ARISING IN ANY WAY OUT OF THE USE OF THIS
% * SOFTWARE, EVEN IF ADVISED OF THE POSSIBILITY OF SUCH DAMAGE.
% *
% ******/
\chapter{Getting Started}

% Getting ztracker
\section{Getting \ztracker{}}
The latest \ztracker{} release can be found at the \ztracker{} web page.
(\weblink{http://ztracker.sourceforge.net/}) 
Releases are named \fname{zt-x.y.zip}, where \fname{x} is the version, and \fname{y}
is the revision.  For each release there is also a source code release, which is
named \fname{zt-x.y-src.zip}.  The source code release is only necessary for 
doing development on \ztracker{}.

% Installing
\section{Installing}
Just unzip the \ztracker{} \fname{.zip} archive to wherever you want it.

\section{Quick Start}
\begin{itemize}
\item Run \fname{zt.exe}.
\item A splash screen will appear.  Click to get past it.
\item You are now in the pattern editor.  \key{CTRL-Tab} switches the view mode.\\
      Press \key{F1} for quick help, \key{F12} for the global configuration, \key{F3}
      for the instrument editor, \key{F5} for play, \key{F8} for stop, and \key{F2} to
      get back to the pattern editor again.
\end{itemize}

\section{Configuring}
Upon installation \ztracker{} is configured for 640x480 resolution running in a window.
This should be adequate for most people.   If you want change the resolution or window
mode, exit \ztracker{} and run the included \fname{ztconf.exe} utility.  This will let
you choose between a number of fixed resolutions, and toggle fullscreen mode.

\section{User Interface}
\notyet{}


% eof
