%/*************************************************************************
% *
% * FILE  introduction.tex
% * $Id: introduction.tex,v 1.3 2001/07/29 22:11:11 tlr Exp $
% *
% * DESCRIPTION 
% *   The introduction chapter.
% *
% * This file is part of ztracker - a tracker-style MIDI sequencer.
% *
% * Copyright (c) 2001, Daniel Kahlin <tlr@users.sourceforge.net>
% * All rights reserved.
% *
% * Redistribution and use in source and binary forms, with or without
% * modification, are permitted provided that the following conditions
% * are met:
% * 1. Redistributions of source code must retain the above copyright
% *    notice, this list of conditions and the following disclaimer.
% * 2. Redistributions in binary form must reproduce the above copyright
% *    notice, this list of conditions and the following disclaimer in the
% *    documentation and/or other materials provided with the distribution.
% * 3. Neither the names of the copyright holders nor the names of their
% *    contributors may be used to endorse or promote products derived 
% *    from this software without specific prior written permission.
% *
% * THIS SOFTWARE IS PROVIDED BY THE COPYRIGHT HOLDERS AND CONTRIBUTORS
% * ``AS IS�� AND ANY EXPRESS OR IMPLIED WARRANTIES, INCLUDING, BUT NOT
% * LIMITED TO, THE IMPLIED WARRANTIES OF MERCHANTABILITY AND FITNESS FOR
% * A PARTICULAR PURPOSE ARE DISCLAIMED.  IN NO EVENT SHALL THE REGENTS OR
% * CONTRIBUTORS BE LIABLE FOR ANY DIRECT, INDIRECT, INCIDENTAL, SPECIAL,
% * EXEMPLARY, OR CONSEQUENTIAL DAMAGES (INCLUDING, BUT NOT LIMITED TO,
% * PROCUREMENT OF SUBSTITUTE GOODS OR SERVICES; LOSS OF USE, DATA, OR
% * PROFITS; OR BUSINESS INTERRUPTION) HOWEVER CAUSED AND ON ANY THEORY OF
% * LIABILITY, WHETHER IN CONTRACT, STRICT LIABILITY, OR TORT (INCLUDING
% * NEGLIGENCE OR OTHERWISE) ARISING IN ANY WAY OUT OF THE USE OF THIS
% * SOFTWARE, EVEN IF ADVISED OF THE POSSIBILITY OF SUCH DAMAGE.
% *
% ******/
\chapter{Introduction}

\ztracker{} is a win32 MIDI only tracker with an interface that is almost a 1:1 clone
of the popular Impulse Tracker DOS tracking software.

It supports multiple MIDI in and out devices, 64 MIDI tracks (expandable to 256),
it can sync to an external sequencer via MIDI clock, \fname{.mid} export, parameter drawing,
96ppqn resolution, and much more.

The original \ztracker{} code (upto 0.82) was written completely by Christopher Micali,
except for the Impulse Tracker loader code, which was written by Austin Luminais.

Version 0.90 of \ztracker{} was the first build to use SDL, the GPLed graphics and input library.  Before
0.90 \ztracker{} used libCON (\weblink{http://www.photoneffect.com/})  \ztracker{} would not have been possible
without libCON.

\section{Goals}
\begin{itemize}
\item a MIDI tracker running under atleast win32
\item fast, simple interface
\item support for multiple out ports
\item well documented
\end{itemize}

\section{Features}
\begin{itemize}
\item near 1:1 copy of Impulse Tracker interface 
\item 64 track sequencer with variable 32-256 rows/pattern, 256 total patterns 
\item easy use of multiple machines across multiple MIDI devices/interfaces 
\item rock solid timing that tested as good as cubase (3/496ppqn error) 
\item load/save compressed \ztracker{} (\fname{.zt}) song files 
\item volume/effect curve drawing in pattern editor 
\item Impulse Tracker song file importing
\item MIDI file (\fname{.mid}) export 
\item auto sync via MIDI-clock 
\item intelligent midi-in with slave to external sync 
\end{itemize}
 
\section{What \ztracker{} is Not}
\ztracker{} is \emph{not} \ldots
\begin{itemize}
\item \ldots A sample player. If you want samples, use a sampling synthesizer or a virtual sampler w/ an
      ASIO card. If we wanted sampling ability, We'd be using buzz.  Buy something nice, like an akai or a yamaha.
\item \ldots A GM composing system.  While you can make GM tunes with zt, that is not what zt is written for.  GM,
      GM2, and XG specific features will not be implemented. 
\end{itemize}

% eof
