%/*************************************************************************
% *
% * FILE  contributing.tex
% * $Id: contributing.tex,v 1.4 2001/09/16 11:13:12 tlr Exp $
% *
% * DESCRIPTION 
% *   The contributing to ztracker appendix.
% *
% * This file is part of ztracker - a tracker-style MIDI sequencer.
% *
% * Copyright (c) 2001, Daniel Kahlin <tlr@users.sourceforge.net>
% * All rights reserved.
% *
% * Redistribution and use in source and binary forms, with or without
% * modification, are permitted provided that the following conditions
% * are met:
% * 1. Redistributions of source code must retain the above copyright
% *    notice, this list of conditions and the following disclaimer.
% * 2. Redistributions in binary form must reproduce the above copyright
% *    notice, this list of conditions and the following disclaimer in the
% *    documentation and/or other materials provided with the distribution.
% * 3. Neither the names of the copyright holders nor the names of their
% *    contributors may be used to endorse or promote products derived 
% *    from this software without specific prior written permission.
% *
% * THIS SOFTWARE IS PROVIDED BY THE COPYRIGHT HOLDERS AND CONTRIBUTORS
% * ``AS IS�� AND ANY EXPRESS OR IMPLIED WARRANTIES, INCLUDING, BUT NOT
% * LIMITED TO, THE IMPLIED WARRANTIES OF MERCHANTABILITY AND FITNESS FOR
% * A PARTICULAR PURPOSE ARE DISCLAIMED.  IN NO EVENT SHALL THE REGENTS OR
% * CONTRIBUTORS BE LIABLE FOR ANY DIRECT, INDIRECT, INCIDENTAL, SPECIAL,
% * EXEMPLARY, OR CONSEQUENTIAL DAMAGES (INCLUDING, BUT NOT LIMITED TO,
% * PROCUREMENT OF SUBSTITUTE GOODS OR SERVICES; LOSS OF USE, DATA, OR
% * PROFITS; OR BUSINESS INTERRUPTION) HOWEVER CAUSED AND ON ANY THEORY OF
% * LIABILITY, WHETHER IN CONTRACT, STRICT LIABILITY, OR TORT (INCLUDING
% * NEGLIGENCE OR OTHERWISE) ARISING IN ANY WAY OUT OF THE USE OF THIS
% * SOFTWARE, EVEN IF ADVISED OF THE POSSIBILITY OF SUCH DAMAGE.
% *
% ******/
\chapter{Contributing to \ztracker{}}

\section{Introduction}
\ztracker{} is an open source project released under the BSD license.  This means you
are very welcome to participate in the development of \ztracker{}.  The license
guarantees that you work on equal terms with the other developers.
The most urgent need is help writing documentation.
Visit the \ztracker{} web site at \weblink{http://ztracker.sourceforge.net/} for
more information.

\section{Bug/request}
There is a bug and request tracking system at
\weblink{http://ztracker.sourceforge.net/}.
Here you can submit bugs, and add suggestions for new features.
You can also send comments, questions, feedback, bugreports, and flames to:
\maillink{ztracker-feedback@lists.sourceforge.net}.

\section{Mailing lists}
\maillink{ztracker-devel@lists.sourceforge.net} is a mailing list for internal
developer discussions.  Feel free to join and read, however non-technical suggestions
should be directed to the \maillink{ztracker-feedback@lists.sourceforge.net} address. 

\section{Making Skins}
Skins are a collection of graphics files and a color description that \ztracker{} uses
for its apperance.
\begin{verbatim}
directory skins\default\
colors.conf
font.fnt
buttons.png
logo.png
load.png
save.png
toolbar.png
about.png
\end{verbatim}
The file \fname{font.fnt} is the font used by \ztracker{}.  To edit this you need
itf 1.65 or a similar font editor.\\
The file \fname{colors.conf} lists the colors that ztracker shall use.
Its format is described in the File Formats chapter.\\
When your \fname{.png}'s have been created you may run \fname{pngcrush} on them to
compress them further.  This does not affect the appearance of the graphics, but
instead just optimizes the way the \fname{.png}'s are stored on disk.
The \fname{default} and \fname{professional} skins were reduced by some 20\%
using this method.

\section{Coding}
\notyet{}

\section{Writing Documentation}
The documentation is written in \LaTeX{}.
Good help on how to write documents in \LaTeX{} can be found in the book
\emph{The \LaTeX{} Companion}~\cite{Goossens1994}, and the freely downloadable  
\emph{The Not So Short Introduction to \LaTeXe{}}~\cite{Oetiker2001}.

If you are writing documentation on a \windows{} machine, we recommend that you
use MiKTeX \weblink{http://www.miktex.org/}, which is an easy to use \LaTeX{}
 distribution.
% eof
