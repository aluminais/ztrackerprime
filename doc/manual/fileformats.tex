%/*************************************************************************
% *
% * FILE  fileformats.tex
% * $Id: fileformats.tex,v 1.13 2001/09/22 20:38:15 tlr Exp $
% *
% * DESCRIPTION 
% *   The File Formats appendix.
% *
% * This file is part of ztracker - a tracker-style MIDI sequencer.
% *
% * Copyright (c) 2001, Daniel Kahlin <tlr@users.sourceforge.net>
% * All rights reserved.
% *
% * Redistribution and use in source and binary forms, with or without
% * modification, are permitted provided that the following conditions
% * are met:
% * 1. Redistributions of source code must retain the above copyright
% *    notice, this list of conditions and the following disclaimer.
% * 2. Redistributions in binary form must reproduce the above copyright
% *    notice, this list of conditions and the following disclaimer in the
% *    documentation and/or other materials provided with the distribution.
% * 3. Neither the names of the copyright holders nor the names of their
% *    contributors may be used to endorse or promote products derived 
% *    from this software without specific prior written permission.
% *
% * THIS SOFTWARE IS PROVIDED BY THE COPYRIGHT HOLDERS AND CONTRIBUTORS
% * ``AS IS�� AND ANY EXPRESS OR IMPLIED WARRANTIES, INCLUDING, BUT NOT
% * LIMITED TO, THE IMPLIED WARRANTIES OF MERCHANTABILITY AND FITNESS FOR
% * A PARTICULAR PURPOSE ARE DISCLAIMED.  IN NO EVENT SHALL THE REGENTS OR
% * CONTRIBUTORS BE LIABLE FOR ANY DIRECT, INDIRECT, INCIDENTAL, SPECIAL,
% * EXEMPLARY, OR CONSEQUENTIAL DAMAGES (INCLUDING, BUT NOT LIMITED TO,
% * PROCUREMENT OF SUBSTITUTE GOODS OR SERVICES; LOSS OF USE, DATA, OR
% * PROFITS; OR BUSINESS INTERRUPTION) HOWEVER CAUSED AND ON ANY THEORY OF
% * LIABILITY, WHETHER IN CONTRACT, STRICT LIABILITY, OR TORT (INCLUDING
% * NEGLIGENCE OR OTHERWISE) ARISING IN ANY WAY OUT OF THE USE OF THIS
% * SOFTWARE, EVEN IF ADVISED OF THE POSSIBILITY OF SUCH DAMAGE.
% *
% ******/
\chapter{File Formats}

\section{Configuration File Formats}
\subsection{The \fname{zt.conf} File Format}
The configuration file \fname{zt.conf} is a text file which contains the settings for
\ztracker{}.  Normally this is updated when exiting \ztracker{} or when using the
\fname{ztconf} utility, however in rare cases you may want to edit it yourself
using a text editor.

\subsubsection{An example \fname{zt.conf}:}
\begin{verbatim}
screen_width: 640
screen_height: 480
default_pattern_length: 128
default_instrument_global_volume: 127
default_highlight_increment: 8
default_lowlight_increment: 8
key_repeat: 30
key_wait: 250
prebuffer_rows: 8
default_view_mode: 3
skin: default
send_panic_on_stop: no
midi_in_sync: no
fullscreen: no
auto_open_midi: yes
step_editing: yes
autoload_ztfile: no
centered_edit: no
autoload_ztfile_filename: autoload.zt
\end{verbatim}

\subsection{The \fname{colors.conf} File Format}
The configuration file \fname{colors.conf} is a text file which contains the color
settings for \ztracker{}.

These entries contain the colors used to draw the beveled \ztracker{} panel.
\begin{verbatim}
Background:     #A49054
Highlight:      #FFDC84
Lowlight:       #504428
\end{verbatim}
%
Text that goes on the \ztracker{} panel and on muted track names.  
\begin{verbatim}
Text:           #000000
\end{verbatim}
%
Text that goes above each track when they are not muted.
\begin{verbatim}
Brighttext:     #CFCFCF
\end{verbatim}
%
Colors used for the information boxes at the top of the screen.
\texttt{Black} is the background color, and \texttt{Data} is the text color.
\begin{verbatim}
Data:           #00FF00
Black:          #000000
\end{verbatim}
%
Colors for the beat display at the bottom left corner of the \ztracker{} window.
\begin{verbatim}
LCDHigh:        #FF0000
LCDMid:         #A00000
LCDLow:         #600000
\end{verbatim}
%
The color of text that is in the pattern editor and all other boxes except
the top info boxes.
\begin{verbatim}
EditText:       #808080
\end{verbatim}
%
The background color of the pattern editor. \texttt{EditBGlow} on lowlight rows,
\texttt{EditBGhigh} on highlight rows, and \texttt{EditBG} everywhere else.
\begin{verbatim}
EditBG:         #000000
EditBGlow:      #14100C
EditBGhigh:     #202014
\end{verbatim}
%
\texttt{SelectedBGLow} is the background color of the pattern editor on a
selected row which is not on a lowlight or highlight row.  If a selected row
is on a lowlight or highlight row, \texttt{SelectedBGHigh} is used instead.
\begin{verbatim}
SelectedBGLow:  #000080
SelectedBGHigh: #0000A8
\end{verbatim}
%
\texttt{CursorRowLow} is the background color of the pattern editor on the row
the cursor is positioned in if it is not a lowlight or highlight row.  If it is
positioned on a lowlight or highlight row, \texttt{CursorRowHigh} is used instead.
\begin{verbatim}
CursorRowLow:   #202020
CursorRowHigh:  #303030
\end{verbatim}

\subsubsection{An example \fname{default.colors.conf}:}
\begin{verbatim}
Background:     #A49054
Highlight:      #FFDC84
Lowlight:       #504428
Text:           #000000
Data:           #00FF00
Black:          #000000
EditText:       #808080
EditBG:         #000000
EditBGlow:      #14100C
EditBGhigh:     #202014
Brighttext:     #CFCFCF
SelectedBGLow:  #000080
SelectedBGHigh: #0000A8
LCDHigh:        #FF0000
LCDMid:         #A00000
LCDLow:         #600000
CursorRowHigh:  #303030
CursorRowLow:   #202020
\end{verbatim}

\section{Song File Format}

The Song File format is the format \ztracker{} uses to store songs on
disk.  \ztracker{} song files have the \fname{.zt} extension.  

\subsection{Current Song Format}
\subsubsection{Song File Header (\texttt{ZThd})}
\begin{verbatim}
0x00  char      chunk_name[4]  (="ZThd")
0x04  u_int32_t chunk_size
0x05  u_int8_t  bpm
0x06  u_int8_t  tpb
0x07  u_int8_t  max_tracks
0x08  u_int8_t  flags
             bit 0: SEND_MIDI_CLOCK
             bit 1: SEND_MIDI_STOP_START
             bit 2: FILE_COMPRESSED
             bit 3: SLIDE_ON_SUBTICK
             bit 4: SLIDE_PRECISION
           bit 5-7: UNDEFINED (always set to zero)
0x09  char      title[26]     
0x23  <next chunk>
\end{verbatim}

\subsubsection{Song File Pattern Lengths (\texttt{ZTpl})}
\begin{verbatim}
0x00  char      chunk_name[4]  (="ZTpl")
0x04  u_int32_t chunk_size     (=256 * sizeof(u_int16_t) = 512)
0x08  u_int16_t pattern_length[256]
0x208 <next chunk>
\end{verbatim}

\subsubsection{Song File Order List (\texttt{ZTol})}
\begin{verbatim}
0x00  char      chunk_name[4]  (="ZTol")
0x04  u_int32_t chunk_size     (=256 * sizeof(u_int16_t) = 512)
0x08  u_int16_t orderlist[256]
0x208 <next chunk>
\end{verbatim}

\subsubsection{Song File Track Mutes (\texttt{ZTtm})}
\begin{verbatim}
0x00  char      chunk_name[4]  (="ZTtm")
0x04  u_int32_t chunk_size     (= max_tracks / 8)
0x08  u_int8_t  track_mutes[max_tracks / 8]
0x08 + chunk_size <next chunk>
\end{verbatim}

\subsubsection{Song File Instrument (\texttt{ZTin})}
\begin{verbatim}
0x00  char      chunk_name[4]  (="ZTin")
0x04  u_int32_t chunk_size
0x08  u_int8_t  inst_number
0x09  u_int16_t bank
0x0b  u_int8_t  patch
0x0c  u_int8_t  midi_device
0x0d  u_int8_t  channel + (flags << 4)
0x0e  u_int8_t  default_volume
0x0f  u_int8_t  global_volume
0x10  u_int16_t default_length
0x12  u_int8_t  transpose
0x13  char title[25]
0x2c  <next chunk>
\end{verbatim}

\subsubsection{Song File Event List (\texttt{ZTev})}
\begin{verbatim}
0x00  char      chunk_name[4]  (="ZTev")
0x04  u_int32_t chunk_size
0x08  <stream of events>
0x08 + chunk_size <next chunk>
\end{verbatim}
The stream of events is encoded as follows:\\
Each event starts with a command byte (\texttt{u\_int8\_t cmd}).
If \texttt{cmd} is between \texttt{0x00} and \texttt{0x3f}, the next value is
an \texttt{u\_int8\_t} (byte event).
If \texttt{cmd} is between \texttt{0x40} and \texttt{0x7f}, the next value is
an \texttt{u\_int16\_t} (word event).
If \texttt{cmd} is between \texttt{0x80} and \texttt{0xbf}, the next value is
an \texttt{u\_int32\_t} (dword event).
If \texttt{cmd} is between \texttt{0xc0} and \texttt{0xff}, the next value is
an \texttt{u\_int16\_t} telling how many string bytes that follow.
(string event)\\

In the current Song File revision only byte events and word events are used.
The data first gets set up. After that the data is inserted
at the appropriate row by issuing the \texttt{0x41} command.
For each new row, only the data that is actually different
from the previous row need to be inserted.

\begin{verbatim}
Byte events:
  0x01 Note
  0x02 Instrument
  0x03 Volume
  0x04 Effect
  0x05 Track (default is initially 0)
  0x06 Pattern (default is initially 0)

Word events:
  0x41 Insert event at row x
  0x42 Length
  0x43 Effect data
\end{verbatim}

\subsection{New Song Format}
This is intended as a new format completely replacing the old format
for introduction somewhere before \ztracker{} 1.0.   Parts of this
specification will under a transitional period be used together
with the old format header.

\subsubsection{Song File Header (\texttt{ZTHD})}
(to be defined)
\begin{verbatim}
0x00  char      chunk_name[4]  (="ZTHD")
0x04  u_int32_t chunk_size
0x08  u_int16_t name_len        | <= short string
0x0a  char      name[name_len]  | (not 0 terminated)
<to be defined>
0x08 + chunk_size <next chunk>
\end{verbatim}

\subsubsection{Song File Song Message (\texttt{SMSG})}
(new as of \ztracker{} 0.94)
\begin{verbatim}
0x00  char      chunk_name[4]  (="SMSG")
0x04  u_int32_t chunk_size
0x08  u_int32_t song_message_len                  | <= long string
0x0c  char      song_message[<song_message_len>]  | (not 0 terminated)
0x08 + chunk_size <next chunk>
\end{verbatim}

\subsubsection{Song File Arpeggio (\texttt{ARPG})}
(new as of \ztracker{} 0.94)
\begin{verbatim}
0x00  char      chunk_name[4]  (="ARPG")
0x04  u_int32_t chunk_size
0x08  u_int16_t number
0x0a  u_int16_t name_len          | <= short string
0x0c  char      name[<name_len>]  | (not 0 terminated)
0x0c  u_int16_t length
0x0c  u_int8_t  num_cc
0x0c  u_int16_t speed
0x0c  u_int16_t repeat_pos
0x0c  u_int8_t  cc[<num_cc>]
0x0c  u_int16_t pitch[<num_entries>]
0x0c  u_int8_t  ccval[<num_cc>][<num_entries>]
0x08 + chunk_size <next chunk>
\end{verbatim}

\subsubsection{Song File Midi Macro (\texttt{MMAC})}
(new as of \ztracker{} 0.94)
\begin{verbatim}
0x00  char      chunk_name[4]  (="MMAC")
0x04  u_int32_t chunk_size
0x08  u_int16_t number
0x0a  u_int16_t name_len          | <= short string
0x0c  char      name[<name_len>]  | (not 0 terminated)
0x0c + name_len  u_int16_t num_entries
0x0e + name_len  u_int16_t mididata[<num_entries>]
0x08 + chunk_size <next chunk>
\end{verbatim}

\subsubsection{Song File Pattern (\texttt{PATT})}
(to be defined)
\begin{verbatim}
0x00  char      chunk_name[4]  (="PATT")
0x04  u_int32_t chunk_size
0x08  u_int16_t number
0x0a  u_int16_t name_len          | <= short string
0x0c  char      name[<name_len>]  | (not 0 terminated)
<to be defined>
0x08 + chunk_size <next chunk>
\end{verbatim}

\subsubsection{Song File Instrument (\texttt{INST})}
(to be defined)
\begin{verbatim}
0x00  char      chunk_name[4]  (="INST")
0x04  u_int32_t chunk_size
0x08  u_int16_t number
0x0a  u_int16_t name_len          | <= short string
0x0c  char      name[<name_len>]  | (not 0 terminated)
<to be defined>
0x08 + chunk_size <next chunk>
\end{verbatim}

\subsubsection{Song File Track Mutes (\texttt{TMUT})}
(to be defined)
\begin{verbatim}
0x00  char      chunk_name[4]  (="TMUT")
0x04  u_int32_t chunk_size
<to be defined>
0x08 + chunk_size <next chunk>
\end{verbatim}

\subsubsection{Song File Order List (\texttt{OLST})}
(to be defined)
\begin{verbatim}
0x00  char      chunk_name[4]  (="OLST")
0x04  u_int32_t chunk_size
0x0c + name_len  u_int16_t num_entries
0x0e + name_len  u_int16_t order[<num_entries>]
0x08 + chunk_size <next chunk>
\end{verbatim}

\section{Skin File Format}
\emph{NOTE: This format is no longer in use.  This section is provided only for the sake of
completeness.} \\
Skin files are containers for a group of files, much like zip archives.  An application
can load a contained file from a skin file by refering to its file name.
In \ztracker{} skin files are use to contain graphics data which \ztracker{} needs.

Each file to be contained in the skin is encoded according to the following scheme and
then concatenated together to form a skin file.  The encoding is split into two parts,
the \emph{header}, and the \emph{payload}. 
All data in the header is stored in little endian order. (i.e the least
significant byte is always first)  

Each entry begins with the header.
First is the name of the contained file. \texttt{namelen} is the length of
the name, and \texttt{name} is the actual name without a trailing 0:
\begin{verbatim}
0x00          u_int32_t namelen
0x04          char name[namelen]
\end{verbatim}
%
Then follows the offset in bytes from the beginning of the skin file to where
the payload begins:
\begin{verbatim}
0x04+namelen  u_int32_t dataoffset           
\end{verbatim}
%
After that follows the uncompressed size, which is the size of the payload will be
after decompression:
\begin{verbatim}
0x08+namelen  u_int32_t realsize
\end{verbatim}
%
Then follows the compressed size, which is the size of the payload before decompression:
\begin{verbatim}
0x0c+namelen  u_int32_t compressedsize
\end{verbatim}
%
After that follows a flag which tells us if the payload is compressed.
If \texttt{compressedflag} is 0 the payload is uncompressed.
If \texttt{compressedflag} is 1 the payload is compressed.
\begin{verbatim}
0x10+namelen  u_int32_t compressedflag
\end{verbatim}
%
Last follows the payload.  If \texttt{compressedflag} is 1 this must be encoded in
a way compliant with the zlib specification algorithm~\cite{ZlibSpec}.
If \texttt{compressedflag} is 0 this is just a copy of the whole file.
\begin{verbatim}
dataoffset    char payload[compressedsize]
\end{verbatim}
%

% eof
